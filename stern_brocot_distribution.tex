\documentclass{article}
\usepackage[pdfusetitle]{hyperref}
\hypersetup{
	pdfkeywords = {Stern-Brocot},
	pdfsubject  = {Stern-Brocot sequences}
}
\title{Number Distribution in the Stern-Brocot Sequence}
\date{}
\author{Reece H. Dunn}

\begin{document}

\maketitle

\section{Stern-Brocot Numbers}

The Stern-Brocot sequence is the numbers:

\begin{displaymath}
[1, 1, 2, 1, 3, 2, 3, 1, 4, 3, ...]
\end{displaymath}

\noindent
This is defined like the Fibbonacci sequence as follows:

\begin{displaymath}
t_0 = 1
\end{displaymath}

\begin{displaymath}
t_1 = 1
\end{displaymath}

\begin{displaymath}
t_k = t_{n-1} + t_n, t_{k+1} = t_n \forall n \in [1, 2, 3, 4, ...]
\end{displaymath}

\noindent
Here, \begin{math}k\end{math} is the last generated term of the sequence.

\noindent
This sequence lists all the rational numbers in the following order:

\begin{displaymath}
\left[
\frac{t_{n-1}}{t_n}
\forall n \in [1, 2, 3, 4, ...]
\right]
=
\left[
\frac{1}{1},
\frac{1}{2},
\frac{2}{1},
\frac{1}{3},
...
\right]
\end{displaymath}

\noindent
This sequence of rational numbers is the same as the Stern-Brocot tree when
using breadth-first traversal (left-to-right, top-to-bottom).

\end{document}
