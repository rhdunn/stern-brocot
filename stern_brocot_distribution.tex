\documentclass{article}
\usepackage[pdfusetitle]{hyperref}
\usepackage{amssymb}
\hypersetup{
	pdfkeywords = {Stern-Brocot},
	pdfsubject  = {Stern-Brocot sequences}
}
\title{Number Distribution in the Stern-Brocot Sequence}
\date{}
\author{Reece H. Dunn}

\begin{document}

\maketitle

\section{Stern-Brocot Numbers}

The Stern-Brocot sequence is the numbers:

\begin{displaymath}
S = [1, 1, 2, 1, 3, 2, 3, 1, 4, 3, ...]
\end{displaymath}

\noindent
This is defined like the Fibbonacci sequence as follows:

\begin{displaymath}
S_0 = 1
\end{displaymath}

\begin{displaymath}
S_1 = 1
\end{displaymath}

\begin{displaymath}
S_k = S_{n-1} + S_n, S_{k+1} = S_n \forall n \in \mathbb{N}_1
\end{displaymath}

\noindent
Here, \begin{math}k\end{math} is the last generated term of the sequence.

\noindent
This sequence lists all the rational numbers in the following order:

\begin{displaymath}
\mathbb{Q}_{>0}
=
\left[
\frac{S_{n-1}}{S_n}
\forall n \in \mathbb{N}_{1}
\right]
=
\left[
\frac{1}{1},
\frac{1}{2},
\frac{2}{1},
\frac{1}{3},
...
\right]
\end{displaymath}

\noindent
This sequence of rational numbers is the same as the Stern-Brocot tree when
using breadth-first traversal (left-to-right, top-to-bottom).

\section{Sequence Partitioning}

When looking into the patterns of the distribution of numbers in the
Stern-Brocot sequence, it is useful to partition the sets generated by those
sequences into partitions that are well defined. This section defines the
mathematics and associated notation used to describe and manipulate these
partitions.

\noindent
Given a sequence:

\begin{displaymath}
A = [a_i \forall i \in \mathbb{N}_{0}]
\end{displaymath}

\noindent
we define the \begin{math}n\end{math}-th \emph{order} partition to be:

\begin{displaymath}
A|^n
=
\left[
[a_{ni+j} \forall j < n : j \in \mathbb{N}_0]
\forall i \in \mathbb{N}_0
\right]
\end{displaymath}

\noindent
for example:

\begin{displaymath}
A|^2
=
[[a_0, a_1], [a_2, a_3], [a_4, a_5], ...]
\end{displaymath}

\noindent
The \begin{math}k\end{math}-th \emph{rank} of this \begin{math}n\end{math}-th
ordered partition is then defined to be:

\begin{displaymath}
A|^n_k
=
\left[
a_{ni+k}
\forall i \in \mathbb{N}_0
\right]
\end{displaymath}

\noindent
for example:

\begin{displaymath}
A|^2_0 = [a_0, a_2, a_4, a_6, ...]
\end{displaymath}

\begin{displaymath}
A|^2_1 = [a_1, a_3, a_5, a_7, ...]
\end{displaymath}

\section{Number Distribution}

Considering the positions at which a given number \begin{math}n\end{math} can
appear in the Stern-Brocot sequence, we can define those positions as the
sequence:

\begin{displaymath}
B^n = [i \forall i \in \mathbb{N}_{1} : S_i = n]
\end{displaymath}

\noindent
and:

\begin{displaymath}
B'^n
= [i+1 \forall i \in \mathbb{N}_{1} : S_i = n]
= [B^n_i + 1 \forall i \in \mathbb{N}_{0}]
\end{displaymath}

\noindent
The distribution of 1s is:

\begin{displaymath}
B'^1
= B'^1|^1_0
= [1, 2, 4, 8, 16, 32, ...]
= [2^i \forall i \in \mathbb{N}_{0}]
\end{displaymath}

\noindent
The distribution of 2s is:

\begin{displaymath}
B'^2
= B'^2|^1_0
= [3, 6, 12, 24, 48, 98, ...]
= [3.2^i \forall i \in \mathbb{N}_{0}]
\end{displaymath}

\noindent
The distribution of 3s is:

\begin{displaymath}
B'^3
= [5, 7, 10, 14, 20, 28, 40, 56, 80, 112, ...]
\end{displaymath}

\begin{displaymath}
B'^3|^2_0
= [5, 10, 20, 40, 80, ...]
= [5.2^i \forall i \in \mathbb{N}_{0}]
\end{displaymath}

\begin{displaymath}
B'^3|^2_1
= [7, 14, 28, 56, 112, ...]
= [7.2^i \forall i \in \mathbb{N}_{0}]
\end{displaymath}

\end{document}
